% $Id: header.tex 1229 2009-10-23 13:58:42Z inf6254 $
%%%%%%%%%%%%%%%%%%%%%%%%%%%%%%%%%%%%%%%%%%%%%%%%%%%%%%%%%%%%%%%%%%%%%%%
\documentclass
  [ %twoside            % beidseitiger Druck
  , openright          % Kapitel beginnen auf einer rechten Seite
  , listof=totoc       % Verzeichnisse im Inhaltsverzeichnis
  , bibliography=totoc % Literaturverzeichnis im Inhaltsverzeichnis
  , parskip=half       % Absätze durch einen vergrößerten Zeilenabstand getrennt
%  , draft              % Entwurfsversion
  ]{scrreprt}          % Dokumentenklasse: KOMA-Script Buch
  

%%%%%%%%%%%%%%%%%%%%%%%%%%%%%%%%%%%%%%%%%%%%%%%%%%%%%%%%%%%%%%%%%%%%%%%
% Packages
%%%%%%%%%%%%%%%%%%%%%%%%%%%%%%%%%%%%%%%%%%%%%%%%%%%%%%%%%%%%%%%%%%%%%%%
\usepackage{scrhack}

 
\usepackage{ifpdf}
\ifpdf
  \usepackage{ae}               % Fonts für pdfLaTeX, falls keine cm-super-Fonts installiert
  \usepackage{microtype}        % optischer Randausgleich, falls pdflatex verwandt
  \usepackage[pdftex]{graphicx} % Grafiken in pdfLaTeX
\else
  \usepackage[dvips]{graphicx}  % Grafiken und normales LaTeX
\fi

\usepackage[a4paper,%
            inner=2.0cm,outer=2.0cm,bindingoffset=0.5cm,%
            top=1.5cm,bottom=1.5cm,%
            footskip=1.0cm,includeheadfoot]{geometry}
            
\usepackage[utf8]{inputenc}         % Input encoding (allow direct use of special characters like "ä")
%\usepackage[english]{babel}
\usepackage[ngerman]{babel}
\usepackage[T1]{fontenc}
\usepackage[automark]{scrpage2} 	 % Schickerer Satzspiegel mit KOMA-Script
\usepackage{setspace}           	 % Allow the modification of the space between lines
\usepackage{booktabs}           	 % Netteres Tabellenlayout
\usepackage{multicol}               % Mehrspaltige Bereiche
\usepackage{quotchap}               % Beautiful chapter decoration
\usepackage[printonlyused]{acronym} % list of acronyms and abbreviations
\usepackage{subfig}                 % allow sub figures
\usepackage{tabularx}               % Tabellen mit fester Breite
\usepackage{lipsum}                 % für Blindtexte


% ------------------- eigene packages ------------------- %

\usepackage{makecell} % fuer Zeilenumbrueche in Tabellenzellen
\usepackage[normalem]{ulem}
\usepackage{tabularx}
\usepackage{calc}
\usepackage[table]{xcolor}
\usepackage{array}
\usepackage{pgfplots}
\usepackage{filecontents}
\usepackage{color}
\usepackage{todonotes}
\usepackage{url}
\usepackage{graphicx}
\usepackage{float}
\usepackage{import} % Noetig fuer verschachtelte Imports 
\usepackage{booktabs} % Tabellenindent
\usepackage{xcolor}
\usepackage{tikz}

\usepackage{csquotes}
\MakeOuterQuote{"} % deutsche Anfuehrungszeichen

\newcolumntype{Y}{>{\centering\arraybackslash}X}

\definecolor{bblue}{HTML}{4F81BD}
\definecolor{rred}{HTML}{C0504D}
\definecolor{ggreen}{HTML}{9BBB59}
\definecolor{ppurple}{HTML}{9F4C7C}


% Layout
\pagestyle{scrheadings}
%\pagestyle{empty}
\clubpenalty = 10000
\widowpenalty = 10000
\displaywidowpenalty = 10000

\makeatletter
\renewcommand{\fps@figure}{htbp}
\makeatother

%% Document properties %%%%%%%%%%%%%%%%%%%%%%%%%%%%%%%%%%%%%%%%%%%%%%%%
\newcommand{\projname}{Serverless Vergleich Spring}
\newcommand{\titel}{Vergleich eines Usecases mit Serverless Technologie gegen\"uber Spring Boot Technologie}
\newcommand{\authorname}{Silas Hoffmann}
\newcommand{\thesisname}{Bachelor Thesis}
\newcommand{\untertitel}{am Beispiel von Instant Payments}
\newcommand{\Datum}{25. August 2021}

\ifpdf
  \usepackage{hyperref}
  \definecolor{darkblue}{rgb}{0,0,.5}
  \hypersetup
  	{ colorlinks=true
  	, breaklinks=true
    , linkcolor=darkblue
    , menucolor=darkblue
    , urlcolor=darkblue
    , citecolor=darkblue
    , pdftitle={\projname -- \untertitel}
    , pdfsubject={\thesisname}
    , pdfauthor={\authorname}
    }
\else
\fi

\bibliographystyle{alpha}

%% Listings %%%%%%%%%%%%%%%%%%%%%%%%%%%%%%%%%%%%%%%%%%%%%%%%%%%%%%%%%
\usepackage{listings}
\KOMAoptions{listof=totoc} % necessary because of scrhack
\renewcommand{\lstlistlistingname}{List of Listings}
\lstset
  { basicstyle=\small\ttfamily
  , breaklines=true
  , captionpos=b
  , showstringspaces=false
  , keywordstyle={}
  }

\lstnewenvironment{inlinehaskell}
{\spacing{1}\lstset{language=haskell,nolol,aboveskip=\bigskipamount}}
{\endspacing}

\lstnewenvironment{inlinexml}
{\spacing{1}\lstset{language=XML,nolol,aboveskip=\bigskipamount}}
{\endspacing}

\newcommand{\haskellinput}[2][]{
  \begin{spacing}{1}
  \lstinputlisting[language=Haskell,nolol,aboveskip=\bigskipamount,#1]{#2}
  \end{spacing}
}

\newcommand{\haskellcode}[2][]{\mylisting[#1,language=Haskell]{#2}}

\newcommand{\mylisting}[2][]{
\begin{spacing}{1}
\lstinputlisting[frame=lines,aboveskip=2\bigskipamount,#1]{#2}
\end{spacing}
}



% --------------- eigene Einstellungen --------------- %

%Quellcodestyle Spezifikationen
\definecolor{DarkPurple}{rgb}{0.4,0.1,0.4}
\definecolor{DarkCyan}{rgb}{0.0,0.5,0.4}
\definecolor{LightLime}{rgb}{0.3,0.5,0.4}
\definecolor{Blue}{rgb}{0.0,0.0,1.0}

% Java Syntaxhighlighting festgelegt
\lstdefinestyle{javaStyle} {
  language=Java, %mit mehreren Sprachen moeglich, ermoeglicht Syntaxhighlighting
  columns=flexible,
  numbers=left,
  frame=single,
  frameround=tttt,
  showstringspaces=false,
  basicstyle=\footnotesize\ttfamily,
  keywordstyle=\bfseries\color{DarkPurple},
  commentstyle=\itshape\color{LightLime},
  stringstyle=\color{Blue}
}

% XML Syntaxhighlighting festgelegt
\lstdefinestyle{xmlStyle} {
  language=XML, %mit mehreren Sprachen moeglich, ermoeglicht Syntaxhighlighting
  columns=flexible,
  numbers=left,
  frame=single,
  frameround=tttt,
  showstringspaces=false,
  basicstyle=\footnotesize\ttfamily,
  keywordstyle=\bfseries\color{DarkPurple},
  commentstyle=\itshape\color{LightLime},
  stringstyle=\color{Blue}
}

% Bash Syntaxhighlighting festgelegt
\lstdefinestyle{bashStyle} {
  language=bash, %mit mehreren Sprachen moeglich, ermoeglicht Syntaxhighlighting
  columns=flexible,
  numbers=left,
  frame=single,
  frameround=tttt,
  showstringspaces=false,
  basicstyle=\footnotesize\ttfamily,
  keywordstyle=\bfseries\color{DarkPurple},
  commentstyle=\itshape\color{LightLime},
  stringstyle=\color{Blue}
}
