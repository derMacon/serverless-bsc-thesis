\chapter{Vorgehensmodell}

\section{Anforderungen an Daten zur Messung des Startup-Verhaltens von Containern}
\begin{itemize}
  \item Problem reduzieren
  \item Messkriterien festlegen
  \item Quellen finden
  \item hier nur theoretisch schwafeln über einen kriterienkatalog und die möglichen „Messwerte“, die man ggf. braucht -> in Kap. 5 dann den Kriterienkatalog konkret aufstellen
\end{itemize}




\section{Anforderungen an Prototypen}
\begin{itemize}
  \item einmal als spring boot und einer weiteren variante mit einer cloud-native technologie -> hier wurde vorgegeben mit serverless zu arbeiten.
\end{itemize}

\subsection{Festlegung fiktiver Workflow}

\subsection{Serverless}
\begin{itemize}
  \item  4.2.2 spring boot 4.2.3 serverless à so bekommt der prof schon beim draufgucken auf das inhaltsverzeichnis die story mit….
\end{itemize}

\section{Anfoderungen an Containerisierungsplattform}
\section{Anforderungen an Lasttest}
\section{Anforderungen Visualiserung und Monitoring zur Unterstützung der Auswertung}

