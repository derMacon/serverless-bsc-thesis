\chapter{Einleitung}

\section{\"Uberblick}
\begin{itemize}
  \item Was gibt es bisher, nur kurz anreissen
  \item wird im Grundlagen Kapitel tiefgehender behandelt...
\end{itemize}

Google, Spotify oder Netflix: Immer mehr Unternehmen setzen heutzutage auf Virtualisierungsplattformen. Aber warum sind Technologien wie zum Beispiel Docker als Containerisierungsplattform, oder Kubernetes als Orchestrator, so beliebt? In der folgenden Thesis möchte ich hierauf eingehen. Ich werden einen Überblick über die verfügbaren Technologien geben und auf diverse Eigenarten eingehen (was heißt es serverless zu deployen etc.). 




\section{Motivation}
\begin{itemize}
  \item Problemstellung kurz beschreiben
  \item DPS m\"ochte untersuchen ...
  \item portierung der anwendung in die cloud (amazon)  Problem ist: spring boot anwendung benötigt viel startup-zeit, heute ist die maximale skalierung in produktion vorbestimmt und muss immer vorgehalten werden, alle server sind 24x7x365 aktive und werden eigentlich nur in 1-2h pro tag in den spitzenzeiten ausgelastet… also nicht nur ineffiziente nutzung von hardware und energie sondern auch von kapital. -> hyperscaler geschäftsmodell verspricht abhilfe…
  \item hier ruhig auch einen Satz zur neuen Sau, die jetzt durchs dorf getrieben wird „new green economy“.. ressourcen sparen um eisbären und gretha thunberg zu retten… energieffizienz…etc bla bla bla sofort-ready vollautomatische Skalierung etc..das sind dann die gganzen Versprechungen der hyperscaler zu der cloud…kannst du hier alle auflisten.  Kunde möchte vorbereitet sein, um ggf. Auflagen der Regulierer zu „green“ erfüllen zu können.
  \item Welche Erkenntnisse sollen gewonnen werden?
  \item Kriterienkatalog nennen
\end{itemize}

Mit den neuen Möglichkeiten die diese neuartigen Technologien mit sich bringen, möchte sich die DPS Gmbh in Zukunft ebenfalls in diese Richtung orientieren, denn die Ansprüche an eine moderne IT-Infrastruktur ändern sich. Man möchte effizienter arbeiten. Nicht nur bezogen auf verbrauchte Energie, sondern zum Beispiel auch auf nötiges Fachpersonal zur Wartung der Artifacts in Produktion. Denn diese Technologien ermöglichen ein Abstraktionslevel, welches es dem Anwendungsentwickler ermöglicht direkt Arbeitspakete für das Deployment zu generieren. Die Grenze zwischen der Operation- und der Entwicklungsabteilung verschwimmen immer mehr und es kommt zu weniger Tradeoffs zwischen mehreren Teams. Auch ermöglicht diese Abstraktion, falls gewünscht, ein komplettes Auslagern der Infrastruktur, wie es zum Beispiel mit Cloud Technologien der Fall ist. Diese Abstraktion ermöglicht den Entwicklern sich mehr auf das Schreiben der Businesslogik zu konzentieren und können durch das Nutzen von Standartisierten Schnittstellen ein vollautomatisiertes Deployment gewährleisten. Diese Prinzip wird je nach Abstraktionstiefe auch als \emph{Infrastructure as Code} (IaC) beziehungsweise \emph{Infrastructure as a Service} (IaaS) bezeichnet. Gerade IaC erlaubt den Entwicklern ihre Programmierkenntnisse auf die Konfiguration von Infrastruktur zu übertragen. Auch die DPS plant in geraumer Zukunft auf Cloud Technologie zu setzen, langfristig ist hierbei eine Portierung in die Cloud von Amazon (AWS) geplant. 

Ein Fokus, welchen ich mir in dieser Thesis im Detail anschauen möchte betrifft das Startup-Verhalten einer Anwendung in genau solch einer Virtualisierten Umgebung. Denn unter den genannten Effizienzgesichtspunkten möchte man heutzutage auch nötige Resourcen vorallem auf Anfrage verwenden und nicht mehr rund um die Uhr laufen lassen, selbst wenn dies zu einem gegebenen Zeitpunkt eigentlich gar nicht nötig wäre. Gerade im Banking Bereich gibt es Zeiträume in denen ein relativ geringer Geldfluss festzustellen ist, während es zu anderen Zeitpunkten zu regelrechten Burst kommen kann, wenn zum Beispiel zu Feiertagen relativ viel Geld den Besitzer wechselt. Um dieses Prinzip der Resourcennutzung auf Anfrage etwas anschaulicher zu gestalten, werde ich eine vereinfachte Kopie einer realen Anwendung vom Unternehmen nachbauen und hinsichtlich der Startzeiten von Containern untersuchen. Der Prototyp beinhaltet mehrere verschiedene Komponenten zur Abarbeitung der Logik, um im Nachhinein auf die Performanz hinsichtlich der genannten Effiziensgesichtspunkte zu untersuchen und zwischen den verwendeten Technologien zu vergleichen.

% Angedacht ist zum Beispiel eine Ihrer größten Anwendungen in die AWS Cloud von Amazon zu deployen. 
% Dazu muss eine Portierung des aktuellen Systems in eine Container Umgebung stattfinden wobei es zu evaluieren gilt, inwiefern die bestehende Codebase dafür überhaupt geeignet ist. Aktuell läuft die Anwendung auf einem JBoss Applikation Server und wurde als Java Enterprise Anwendung konzipiert. Java ist jedoch schon fast für seine relativ langen Initialisierungsphasen der einzelnen Komponenten bekannt. Gerade in einer Containerisierten Anwendung wird viel auf kleinere abgeschlossene logische Einheiten gesetzt, die es ermöglichen, das Systsem möglichst variabel und skalierbar zu gestalten. Hierbei stößt die monolithische Struktur des aktuellen Ansatzes wahrscheinlich an seine Grenzen. Um auch langfristig 

% möchte, werde ich einen Prototypen vorstellen, welcher ein vereinfachtes Abbild einer bestehenden Anwendung in Produktion darstellt im Hintergrund jedoch auf gerade diese neuartigen Technologien setzt. Der besondere 