\chapter{Einleitung}

\section{\"Uberblick}
\begin{itemize}
  \item Was gibt es bisher, nur kurz anreissen
  \item wird im Grundlagen Kapitel tiefgehender behandelt...
\end{itemize}

\section{Motivation}
\begin{itemize}
  \item Problemstellung kurz beschreiben
  \item DPS m\"ochte untersuchen ...
  \item portierung der anwendung in die cloud (amazon)  Problem ist: spring boot anwendung benötigt viel startup-zeit, heute ist die maximale skalierung in produktion vorbestimmt und muss immer vorgehalten werden, alle server sind 24x7x365 aktive und werden eigentlich nur in 1-2h pro tag in den spitzenzeiten ausgelastet… also nicht nur ineffiziente nutzung von hardware und energie sondern auch von kapital. -> hyperscaler geschäftsmodell verspricht abhilfe…
  \item hier ruhig auch einen Satz zur neuen Sau, die jetzt durchs dorf getrieben wird „new green economy“.. ressourcen sparen um eisbären und gretha thunberg zu retten… energieffizienz…etc bla bla bla sofort-ready vollautomatische Skalierung etc..das sind dann die gganzen Versprechungen der hyperscaler zu der cloud…kannst du hier alle auflisten.  Kunde möchte vorbereitet sein, um ggf. Auflagen der Regulierer zu „green“ erfüllen zu können.
  \item Welche Erkenntnisse sollen gewonnen werden?
  \item Kriterienkatalog nennen
\end{itemize}
