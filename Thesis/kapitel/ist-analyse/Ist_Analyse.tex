\chapter{Ist-Analyse}

\section{JBoss (Microprofile)}
\begin{itemize}
  \item aktuelle Architektur beschreiben
  \item Hinweis darauf geben, dass Prototyp in der Thesis vereinfacht mit Spring dargestellt wird
  \item jetzt ja fiktiv SpringBoot, hier rein technischer Ist-Stand, Systemarchitektur….
\end{itemize}

Die aktuelle Verarbeitung von Payments innerhalb der Anwendung läuft in Produktion auf vier Instanzen des kommerziellen Applikation-Servers "\emph{JBoss}". Im Development wird hierbei eine Open-Source Variante namens "\emph{Wildfly}" verwendet. In diesen Application Servern werden entsprechende .war Dateien deployed welche den ausführbaren Code der Anwendung beinhalten. Der Application Server bietet dem Server-Teil der Client-Server-Anwendung eine Laufzeitumgebung, in der dieser ausgeführt werden kann. JBoss bietet nun standartisierte Schnittstellen nach dem jeweiligen Java Enterprise Standard um zum Beispiel die Kommunikation mit der Außenwelt zu ermöglichen oder um der Anwendung eine Persistenzschicht zur Verfügung zu stellen. Um eine gleichmäßige Aufteilung der Last zu gewährleisten, teilt ein so genannter "\emph{Load Balancer}" die eingehenden Nachrichten den entsprechenden Instanzen zu. Jede der vorhandenen Instanzen besitzt eine minimale sowie maximale Anzahl an parallel ausführbaren Prozessen. Diese Angaben werden auch "\emph{max. / min. Poolsize}" genannt. Eine minimale Poolsize muss gegeben sein, um sicherzustellen, dass eine gewisse Grundlast falls nötig sofort bearbeitet werden kann, daher darf diese minimale Anzahl auch nicht Null betragen. Die maximale Poolsize stellt sicher, dass es zu keiner Überlastung des Systems kommt. Wenn eine Instanz bereits mit der maximale Anzahl an Prozessen arbeitet, wird dies dem Loadbalancer signalisiert und dieser teilt der entsprechenden Komponente in diesem Zeitraum keine weiteren Nachrichten mehr zu. Um zu gewährleisten, dass die Nachrichten nicht verloren gehen, werden sie in eine "\emph{Request Queue}" geschrieben, welche lediglich dazu gedacht ist den Overhead abzuspeichern. Wie die Daten im Detail verarbeitet werden, ist für die weitere Betrachtung irrelevant und wird daher nicht weiter erläutert.

\todo{Simples Schaubild einfuegen}


\section{Probleme}
\begin{itemize}
  \item Probleme mit aktuellem System (Stichwort Deployment, Wartbarkeit)
  \item Prof. hat extra darauf hingewiesen, dass es nicht nur um die Vorteile der Cloud gehen soll
  \item „Erwartete Probleme in einer Cloud Umgebung“ à aktuell „..starre Hardware und Software- Skalierung…“ unerwartete lastspitzen könnten zu problemen führen, wenn sie die erwartete und verfügbare obergrenze an kapazität übersteigt und wie oben gesagt ineffiziente nutzung von kapital….nochmal mit anderen worten aus 
  \item  1.2 à die auflistung der probleme hier, müssen dann in der zusammenfassung wieder auftauchen und abgehakt werden! Die wollen wir ja auch lösen…
\end{itemize}
