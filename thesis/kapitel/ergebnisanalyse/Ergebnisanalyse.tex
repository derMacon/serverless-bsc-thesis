\chapter{Ergebnisanalyse}

\todo{nochmal schauen ob Überleitung noch stimmt wenn Implementierungsteil geschrieben wurde...}
\todo{Verweis auf Anhang mit allen Tabellen für Metrikinformationen}


\begin{wraptable}{r}{5.8cm}
  \centering
  \label{tab:latency}
  \hspace{1cm}
  \begin{tabular}{@{}lc@{}}
    \toprule
    Metrik & Dauer \\
    \midrule
    Node.js \\
    \hspace{3mm}Gesamtdurchschnitt & 38416 \\
    \hspace{3mm}Nachrichteneingang & 35027 \\
    \hspace{3mm}Verarbeitungsdauer & 3388 \\
    \midrule
    Spring Boot \\
    \hspace{3mm}Gesamtdurchschnitt & 61800 \\
    \hspace{3mm}Nachrichteneingang & 58771 \\
    \hspace{3mm}Verarbeitungsdauer & 3029 \\
    \bottomrule
  \end{tabular}
  \caption[Latenzzeit - Vergleich]{Latenzzeiten}
\end{wraptable} 


Nachdem im letzten Abschnitt ausführlich beschrieben wurde wie der komplette Komponentenstack im Detail implementiert wurde, werden im Folgenden die erhaltenen Metriken erläutert. Es wird insbesondere auf die Bedeutung einzelner Messwerte sowie auf mögliche Begründungen dieser eingegangen. Die genauen Messwerte wurden im Anhang aufgelistet und sind ebenfalls unter folgender Url einsehbar: \url{https://github.com/derMacon/serverless-bsc-thesis/tree/main/data} \todo{am Ende nochmal nachschauen ob URL weiterhin stimmt}

\todo{Messwerte im Anhang noch einmal gesammelt darstellen}

\section{Ergebnisse}

% \begin{itemize}
%   \item Vorstellung der erhaltenen Daten
%   \item Interpretation / Analyse allerdings Teil vom naechsten Kapitel?
% \end{itemize}


Für die Messung der folgenden Ergebnisse wurden in erster Linie zwei Lastentests durchgeführt. Der Erste bezog sich auf eine Skalierung als Reaktion auf Payment-Nachrichten, die vom System empfangen wurden. Die Zweite Messung erfolgte zu direkt veranlassten Skalierungsschritten über eine gesonderte Schnittstelle vom Skalierer-Proxy-Service, die komplett auf eingehende Payment-Nachrichten verzichtet dadurch allerdings auch das Stufenmodell\footnote{siehe Abschnitt \ref{}} \todo{Referenz in Footnote einbinden}. Diese zweite Messung ermöglichte erst die genaue Übersicht über die Containeranzahl im Bezug zur Skalierungszeit (siehe Abbildung \ref{fig:specContainers}).


\subsection{Latenzzeit \checkmark}
Die wichtigsten Messdaten bezüglich der Leistungsfähigkeit des Systems beziehen sich auf den Datendurchsatz beziehungsweise die entsprechende Latenzzeit. In der Tabelle \ref{tab:latency} wurden diese für die beiden betrachteten Backendtechnologien gegenübergestellt. Alle Angaben wurden in Millisekunden erfasst. Das System benötigt zur Bearbeitung einer eingegangenen Nachricht durch einen Node.js Konsumenten im Schnitt 38.4 Sekunden, während sich dieser Wert bei Spring Boot auf 61.8 beläuft. Diese Werte bilden lediglich einen Durchschnitt aller erhaltenen Metriken ab, eine genauere Aufteilung bezüglich des Zusammenhangs zu den Skalierungsgrößen erfolgt im nächsten Abschnitt (siehe \todo{Ref einfuegen}). Der Gesamtdurchschnitt setzt sich aus zwei Zeitangaben der Pipeline zusammen: 

\begin{enumerate}
  \item Nachrichteneingang: Diese Metrik beschreibt den Zeitraum zwischen erhaltener Anfrage im System und Acknowledgement durch die Konsumer-Komponente, dass die Nachricht nun bearbeitet wird. Sie beläuft sich bei der Node.js Komponente auf ungefähr 35.0 Sekunden und 58.8 Sekunden bei der Spring-Boot-Komponente. 
  \item Verarbeitungsdauer: Diese Metrik beschreibt den Zeitraum zwischen Acknowledgement durch den Konsumer, dass die Nachricht erhalten wurde, sowie dem Abspeichern des extrahierten Wertes durch den Konsumenten in der Datenbank. Dieser Werte beläuft sich bei beiden Implementierungen auf etwas über drei Sekunden. Bei dieser Metrik liegt der Fokus allerdings auf dem minimalen Zeitunterschied im Millisekundenbereich, da in beiden Implementierungen eine künstliche Verlangsamung (\emph{Sleep-Funktion}) eingebaut wurde um die erhaltenen Messwerte besser nachvollziehen zu können. Bei Node.js liegt die tatsächliche Verarbeitungsdauer bei 388 Millisekunden \emph{(total 3388)}, während sie bei Spring Boot bei 29 Millisekunden liegt \emph{(total 3029)}.
\end{enumerate}


\subsection{Skalierungsdauer \checkmark}
Diese Gruppe von Metriken bezieht sich auf die genauen Zeiträume, welche benötigt werden, um einen Container mit der entsprechenden Implementierung hochzufahren. Die ermittelten Zeiten stellen im Folgenden die durchschnittlichen Initialisierungszeiten eines einzelnen Containers dar. Sie beziehen sich nicht auf die Gesamtdauer der Initialisierungsphase aller beteiligten Container.


\paragraph{Aufschlüsselung nach Services \checkmark}
Die abstrakteste Metrik bezieht sich auf den durchschnittlichen Zeitraum zum Initialisieren der Container eines bestimmten Services. Hierbei wird zwischen der Node.js sowie Spring-Boot-Komponente zum Konsumieren der Eingangsnachrichten unterschieden. Die Container, welche eine Node.js-Instanz beinhalten, benötigen im Schnitt 6.6 Sekunden zum Hochfahren, während diejenigen mit einer Spring-Instanz 35.3 Sekunden brauchen. Diese verhältnismäßig großen Zeiträume lassen sich unter anderem auf die Anzahl der hochgefahrenen Instanzen zurückführen. Eine genauere Aufschlüsselung dieses Prozesses erfolgt in den nächsten Abschnitten. 


\begin{figure}
	\centering
	\begin{minipage}[b]{0.4\textwidth}
		\begin{tikzpicture}
			\begin{axis}[
					width  = 0.8\textwidth,
					height = 8cm,
					major x tick style = transparent,
					ybar=2*\pgflinewidth,
					bar width=14pt,
					ymajorgrids = true,
					xlabel = {Service},
					ylabel = {{\o} Initialisierungsgeschw.},
					symbolic x coords={Node.js,Spring Boot},
					xtick = data,
					scaled y ticks = false,
					enlarge x limits=0.25,
					ymin=0,
					legend cell align=left,
					legend style={
						at={(1,1.05)},
						anchor=south east,
						column sep=1ex
					}
				]
				\addplot[style={bblue,fill=bblue,mark=none}]
				coordinates {(Node.js,6.610) (Spring Boot,35.336)};
			\end{axis}
		\end{tikzpicture}
		\caption[Startzeit Container - Service]{Serviceübersicht}
		\label{fig:specContainers}
	\end{minipage}%
	\begin{minipage}[b]{0.6\textwidth}
		\begin{tikzpicture}
			\begin{axis}[
					width  = 0.9\textwidth,
					height = 8cm,
					major x tick style = transparent,
					ybar=2*\pgflinewidth,
					bar width=14pt,
					ymajorgrids = true,
					xlabel = {Skalierungsstufe},
					ylabel = {{\o} Initialisierungsgeschw.},
					symbolic x coords={Stufe1,Stufe2,Stufe3},
					xtick = data,
					scaled y ticks = false,
					enlarge x limits=0.25,
					ymin=0,
					legend cell align=left,
					legend style={
						at={(1,1.05)},
						anchor=south east,
						column sep=1ex
					}
				]
				\addplot[style={bblue,fill=bblue,mark=none}]
				coordinates {(Stufe1,2.976) (Stufe2,3.892) (Stufe3,6.898)};
				
				\addplot[style={ggreen,fill=ggreen,mark=none}]
				coordinates {(Stufe1,8.056) (Stufe2,16.019) (Stufe3,36.582)};
				
				\legend{Node.js,Spring Boot}
			\end{axis}
		\end{tikzpicture}
		\caption[Startzeit Container - Stufenweise]{Stufenübersicht}
		\label{fig:specContainers}
	\end{minipage}
\end{figure}


\paragraph{Aufschlüsselung nach Skalierungsstufe \checkmark}
Um einen beispielhaften Skalierungsalgorithmus zu implementieren, wurde ein Regelsatz verfasst, der vom Alert-Manager zur Laufzeit automatisch in einem festgelegten Intervall ausgewertet wird. Um diesen Regelsatz zu Vorführungszwecken nicht unnötig ausführlich zu gestalten, wurde das Hinzufügen neuer Instanzen in Stufen organisiert. 

Im durchgeführten Testszenario betrug die Grenze zum Überschreiten der ersten Stufe eine Anzahl von 15, für die zweite Stufe 30 und für die dritte Stufe 100 unbeantwortete Nachrichten in der jeweiligen Warteschlange. Wenn im Folgenden von einem "\emph{Burst}" gesprochen wird, ist hiermit das Überschreiten einer dieser Grenzen gemeint. Bei dem kleinstmöglichen Burst bezüglich der Warteschlange der Node.js-Komponente, nimmt das System im Schnitt 2.9 Sekunden zum Starten der Container in Anspruch um der Arbeitslast nach Skalierung mit insgesamt neuen 5 Containern-Instanzen zu begegnen. Bei der Spring-Boot-Komponente wurde hierfür ein Wert von 8.1 Sekunden gemessen. Im mittleren Grenzintervall wurde bezüglich Node.js eine Initialisierungsdauer von 3.8 Sekunden und aufseiten von Spring 16.0 Sekunden festgestellt um nach der Skalierung mit 10 Instanzen zu arbeiten. Beim höchsten Burst werden kam es zu einer bei Node.js zu einer Initialisierungsphase von 6.9 Sekunden und bezüglich Spring zu 36.6 Sekunden. Hierbei sollen am Ende 30 Containerinstanzen des Konsumenten einsatzbereit sein. Da beim Skalierungsalgorithmus mit den beschriebenen Stufen gearbeitet wird, kann hierbei nicht genau gesagt werden, wie viele Container tatsächlich im Endeffekt initialisiert wurden, da stets mit den Differenzen zwischen den Stufen gearbeitet wird. Wenn zum Beispiel bereits \emph{n} Container laufen und neue Nachrichten einen Burst verursachen, sollen nur die benötigten Instanzen kreiert werden, um genau die fehlende Anzahl zu decken. Die genaue Übersicht nach parallelen Instanziierungen wurde im Paragraphen (siehe Abschnitt \ref{par:specContainer} \todo{ref pruefen}) genauer beschrieben.


\paragraph{Aufschlüsselung nach Containeranzahl \checkmark}
\label{par:specContainer}
Die grundlegendste Metrik beschreibt die Zeiten aufgeschlüsselt nach Anzahl gleichzeitig hochfahrender Container (siehe \ref{fig:specContainers}). Wie im Graphen erkennbar, handelt es sich um ein lineares Wachstum. Mit jedem zusätzlichen zeitgleich erstellten Container, dauert der Initialisierungsprozess bei der Spring-Implementierung im Schnitt 1611 Millisekunden länger. Bei der Node.js Implementierung liegt dieser Wert bei 194 Millisekunden. So dauert das Initialisieren eines einzelnen neuen Containers bei der Spring-Boot-Komponente 4,8 Sekunden während sich dieser Wert bei der Node.js-Koponente lediglich 2,6 Sekunden beläuft. Um diese nach Containeranzahl aufgeschlüsselten Werte zu erhalten, wurde nicht wie bei den anderen Skalierungstests auf die öffentliche Schnittstelle mittels Nachrichtengenerierung zurückgegriffen, es wurde stattdessen eine interne Schnittstelle mittels dediziertem Skript verwendet um möglichst störungsfrei direkt auf die Komponente, die das Skalieren orchestriert zuzugreifen. Die nachfolgenden Metriken wurden allerdings durch die öffentliche Schnittstelle generiert.

\begin{wrapfigure}{l}{0.55\textwidth}
  \centering
  \caption[Startzeit Container - Anzahl spezifisch]{Anzahlspezifisch}
  \label{fig:specContainers}
  \begin{tikzpicture}
    \begin{axis}[xlabel={Zusätzlich hochgefahrende Container}, ylabel={Startzeit}]
      \addplot table [x=additionalCnt, y=startupTime, col sep=comma] {kapitel/ergebnisanalyse/_data/springSpecificBenchmarks.csv};
      \addlegendentry{Spring Boot}
      \addplot table [x=additionalCnt, y=startupTime, col sep=comma] {kapitel/ergebnisanalyse/_data/nodeSpecificBenchmarks.csv};
      \addlegendentry{Node.js}
    \end{axis}
  \end{tikzpicture}
\end{wrapfigure}


\paragraph{Zeitliche Aufschlüsselung \checkmark}
Wie in Bild \ref{fig:grafanaScreenshot01} zu sehen, wurde diverse Panels zur zeitlichen Übersicht bereitgestellt. Diese wurden jedoch vor allem zu Kontrollzwecken implementiert. Diese visuelle Darstellung stellte sich während der Entwicklungszeit als sehr hilfreich dar, um Fehler frühzeitig zu erkennen und beheben zu können. Für die Auswertung der Ergebnisse haben diese allerdings wenig Relevanz. Der Vollständigkeit halber soll an dieser Stelle dennoch eine kurze Zusammenfassung der zeitbasierten Messwerte erfolgen. Für alle bisher behandelten Metriken gibt es Panels im Grafana Dashboard. Da die zeitliche Erfassung in Grafana selbst allerdings intervallbasierte Abfragen stellt, sind diese im Vergleich zu den Werten, welche in der Datenbank durch das System generiert und abgelegt wurden, sehr ungenau. Selbstverständlich wurde in der bisherigen Analyse ausschließlich auf die persistierten Daten zurückgegriffen. Bezüglich der Konsumenten wurde neben dem zusammengefassten durchschnittlichen Startverhalten auch der jeweils aktuelle Initialisierungszeitpunkt dargestellt. Hierbei war eine klare Korrelation zwischen der Anzahl der parallel startenden Konsumenten mit der erhöhten Initialisierungsdauer zu beobachten. Außerdem wurden diverse Metriken aus dem Activemq-Broker selbst ausgelesen. So ist es zum Beispiel möglich nachzuvollziehen, wie viele Nachrichten sich zu einem gegebenen Zeitpunkt nicht nur in einer bestimmten Warteschlange sondern auch innerhalb des ganzen Systems befinden. Außerdem wurde die zeitliche Abfolge zwischen einem Burst an unbeantworteten Nachrichten und dem Anstieg der Containerinstanzen in einem Dashboard-Panel dargestellt.

\begin{figure}[ht!]
	\centering
	\includegraphics[width=\linewidth]{kapitel/problemloesung/implementierung/_img/grafana-dashboard-01}
	\caption[]{Grafana Dashboard}
	\label{fig:grafanaScreenshot01}
\end{figure}


\section{Analyse}

% \begin{itemize}
%   \item Interpretation / Analyse der Daten
%   \item Begruendung fuer Verhalten suchen
% \end{itemize}

\subsection{Latenzzeit}
Hinsichtlich der Latenzzeit überrascht vor allem die deutliche Diskrepanz zwischen der Initialisierungsdauer der unterschiedlichen Technologien. Durch die Unterteilung der Pipeline in die zwei Messwerte ist erkennbar, dass diese auf die zeitliche Dauer bis zum Nachrichteneingang zurückzuführen ist. Hierbei läuft die Dauer der Initialisierungsphasen weit auseinander (siehe \ref{ss:skalierungsdauer}).

Bezüglich der Verarbeitungsgeschwindigkeit ist die Spring-Boot-Komponente jedoch im Vergleich schneller. Dies ist auf die Natur einer kompilierten Sprache zurückzuführen. Java Code wird im Vorwege in entsprechenden Bytecode übersetzt, während eine Skriptsprache wie Javascript innerhalb der Node.js-Komponente zur Laufzeit übersetzt wird. Da in den Komponenten nur minimale Logik verbaut wurde, ist der festgestellte zeitliche Unterschied in der Bearbeitungsdauer mit 359  Millisekunden zwar nicht so gravierend wie der Unterschied hinsichtlich der Initialisierungsphase der Container, allerdings drastischer als vorher angenommen. Da der Prototyp eine sehr einfache Ausführungslogik implementiert, ist dieser Messwert doch relativ hoch. In Anbetracht der Komplexität der realen Banking-Anwendung gilt es ein spezifisches Konzept für das Regelwerk zum Starten der Container zu entwerfen. Hierbei muss eine Untersuchung bezüglich der Rentabilität eines neuen startenden Containers gegenüber eines laufenden Containers evaluiert werden.


\subsection{Skalierungsdauer}
\label{ss:skalierungsdauer}
Die erheblich abweichende Initialisierungsgeschwindigkeit der beiden unterschiedlichen Komponenten lässt sich im Kern auf auf die Initialisierungs des Spring-Containers sowie des dazugehörgien Application-Contexts zurückführen. Das Framework bietet durch die ... \todo{Weiterschreiben, schauen ob es diese Erklärung schon irgendwo in der Thesis gibt.}
Kern des Spring Frameworks ist der \emph{Spring Container}. Dieser verwaltet fachliche und nichtfachliche Objekte, die eine Anwendung ausmachen. Die verwalteten Objekte werden als \emph{Beans} bezeichnet. "Eine Bean ist ein Objekt, das vom Spring-Container instanziiert und konfiguriert wurde und dessen Lebenszyklus vom Container verwaltet wird. Die Abhängigkeiten zwischen Beans sind als Metadaten im Container verfügbar" \cite[Kapitel~3.1.1]{simons-spring}. 


\section{Diskussion}

\subsection{Begr\"undung Startupzeit}
\begin{itemize}
  \item Warum Node.js schneller ist
\end{itemize}

\begin{itemize}
  \item Erlaeutern warum die erhaltenen Ergebnisse in einem real-life Szenario vielleicht nicht aussagekraeftig sein koennten
\end{itemize}
