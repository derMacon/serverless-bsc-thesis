\chapter{Zusammenfassung}

Nachdem im letzten Abschnitt sämtliche Messwerte vorgestellt, analysiert und beurteilt wurden, folgt nun eine Zusammenfassung der gesamten Thesis.

Die DPS Engineering Gmbh möchte in Zukunft auf moderne Technologien setzen, um ihre Plattform für die Echtzeitüberweisungen performanter und ressourcensparender zu gestalten. Hierzu wurde eine kurzer Einblick in die zugrunde liegende Motivation gegeben. Von besonderer Relevanz ist hierbei die \emph{Cloudfähigkeit} im Bezug auf die Initialisierungsgeschwindigkeit der zu evaluierenden Technologien. Damit ein sachlicher Überblick gegeben werden kann, wurden die zu untersuchenden Ziele mithilfe einer im Detail beschriebenen ISO Norm festgelegt. Dabei wurde sich nicht nur auf die Erklärung der einzelnen Begriffe beschränkt, sie wurden ebenfalls nach Relevanz für die Kernziele geordnet und in Gruppen sortiert.

Aus den ermittelten Normen von Qualitätsmerkmalen wurden in einem weiteren Schritt mögliche zu ermittelnde Metriken festgelegt. Für diese Metriken wurden an dieser Stelle bereits eine erste Überlegung dargestellt inwiefern sie mithilfe eines zu implementierenden Prototypen ermittelt werden können. Es wurde außerdem auf das benötigte Tooling und die Funktionsweise der Messdatenerhebung eingegangen. 

Anschließend wurden die Anforderungen an den Prototypen in verschiedenen Schritten dargestellt. In einem ersten Schritt wurde ein fiktiver Workflow dargestellt, welcher ein einfaches Abbild der real existierenden Logik des eingesetzten Softwareprodukts der DPS darstellt. Darauf folgend wurden die Anforderungen an die artefaktbasierten sowie skriptbasierten Technologien beschrieben. Der Einsatz dieser beiden Gegensätze erlaubt eine Gegenüberstellung und somit eine Evaluierung, welche der beiden Technologien demnach am besten für den beschriebenen Usecase geeignet ist. Neben den Anforderungen an den Prototypen wurden ebenfalls sämtliche Anforderungen an die Deploymentplattform sowie den Lasttest und die dazugehörige Visualisierung / Monitoring definiert. 

Nachdem die nötigen Anforderungen an das System im Detail erläutert wurden, erfolgte ein Auswahlprozess der im Vorwege erläuterten Metriken des Start-up-Verhaltens von Contain-Instanzen. Im Anschluss daran folgte die letztendliche Beschreibung der Implementierung des Prototypen. Hierbei wurde sowohl auf die generelle Architektur der Komponenten sowie der verwendeten Technologie eingegangen, als auch die selbst implementierte Logikstruktur näher erläutert. Hiernach wurde das Deployment auf der Containerplattform sowie die Implementierung des Lasttests beschrieben.


Im Anschluss an die Beschreibung der Funktionsweise des Prototypen folgte die Vorstellung der Ergebnisse der Messungen hinsichtlich des Lasttests. Die Messewerte wurden im Detail erläutert und anschließend analysiert. Außerdem wurden diese kritisch hinterfragt und hinsichtlich der Eignung zur Übertragbarkeit auf einen realen Anwendungsfall beurteilt. Danach wurden erlangten Ergebnisse hinsichtlich der Kernziele der Thesis untersucht und es wurde ein Fazit zur Eignung der untersuchten Technologien für den beschriebenen Usecase des Unternehmens erlangt.
